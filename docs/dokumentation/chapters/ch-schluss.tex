\chapter{Zusammenfassung und Ausblick}
\label{sec:schluss}

Nach einer kurzen Evaluation von verf�gbaren Werkzeugen zur Analyse von File-IO wurden diese als unzureichend f�r den angestrebten Zweck beurteilt. Lediglich Darshan erf�llt die grundlegenden Anforderungen. Mit Darshan k�nnen POSIX-IO und MPI-IO in eine Log-Datei protokolliert werden. Diese Datei kann dann durch ein Analyseprogramm augewertet werden. Darshan unterst�tzt jedoch nur POSIX-IO und MPI-IO. Andere Arten von Dateizugriffen werden nicht ber�cksichtigt. Diese sind allerdings im HPC-Bereich von entscheidender Bedeutung. So erfordert eine hochparallelisierte Kommunikation zu Dateien zum Beispiel ein Dateisystem wie Lustre. Dieses bringt entsprechend eigene File-IO-Funktionen mit sich, welche �ber Darshan nicht protokolliert werden k�nnen. Zudem schreibt Darshan die Protokolldateien in einem Bin�rformat. Dieses ist entsprechend seiner Natur nicht leicht aus anderen Anwendungen heraus zu nutzen. Insbesondere bei der Verwendung von Programmiersprachen welche die bin�re Kodierung vor dem Entwickler verbergen (wie zum Beispiel Java) stellt das Bin�rformat ein unerw�nschtes Hindernis dar.

Nach der Evaluation wurde daher mit dem Entwurf einer eigenen L�sung begonnen. Dabei lag der Fokus im ersten Schritt auf den Wrappern. Deren Architektur und das Datenformat der Protokollierung wurden nach Gesichtspunkten der Performance und der Nutz- und Erweiterbarkeit gew�hlt. Dabei ergaben sich f�r das Abfangen der File-IO-Funktionsaufrufe �hnliche Ans�tze wie in Darshan, w�hrend die Protokollierung andere Ans�tze verfolgt. Neben der Architektur der Wrapper wurde f�r das gesamte System eine erste grobe Architektur festgelegt.

Welche Daten durch die ersten Wrapper f�r einfache Analysen gebraucht werden, wurde durch eine Betrachtung m�glicher File-IO-Konstellatione ermittelt. Abh�ngig von den k�nftig im Projekt geplanten Analysen m�ssen jedoch weitere Daten protokolliert werden. Diese Erkenntnis wurde beim Entwurf der Wrapper und insbesondere bei der Wahl des Datenformats ber�cksichtigt.

Als Datenformat f�r die Protokollierung wurde JSON gew�hlt. Dieses Format erm�glicht eine Enfache Nutzung der protokollierten Daten mittels unterschiedlicher Programmiersprachen aus verschiedenen Plattformen heraus. Dabei h�lt sich der Overhead beim Protokollieren im Vergleich zu anderen Formaten in Grenze. JSON bietet zudem die M�glichkeit die Protokollierung einfach zu erweitern. Werden nachtr�glich durch Wrapper zus�tzliche Daten ben�tigt, so k�nnen diese im JSON-Format hinzugef�gt werden, ohne dass dies ein Auslesen der Datei beeintr�chtigt.

Nach einem erfolgreichen Test des Konzepts der Wrapper wurde mittlerweile mit der Implementierung begonnen. Dabei liegt der Fokus zun�chst auf POSIX-IO und MPI-IO. Im kommenden Semester ist geplant die Wrapper weitgehend fertig zu stellen. Nach erfolgreicher Implementierung der Wrapper wird mit der Umsetzung der restlichen Komponenten begonnen.

Da bei dem Entwurf einer eigenen L�sung f�r die Wrapper zum Protokollieren des File-IOs �hnliche Ans�tze wie in Darshan gew�hlt wurden, ist eine weitergehende Analyse von Darshan bez�glich der Performance sinnvoll. Hier wird insbesondere ein Vergleich zwischen der im Rahmen dieses Projektes erstellten libiotrace und Darhans libdarshan.so angestrebt. Mit dieser Aufgabe kann begonnen werden, sobald die Wrapper einen mit Darshan vergleichbaren Funktionsumfang erreicht haben.