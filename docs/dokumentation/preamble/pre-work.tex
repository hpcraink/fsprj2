% ------------------------------------------------------------------------
% LaTeX - Preambel ******************************************************
% ------------------------------------------------------------------------
% pre-work
% ========================================================================
% % ToDo kennzeichnen
\newcommand{\workTodo}[1]{\textcolor{red}{todo: #1}}

% % F�r Datum und Zeit in Fusszeile
% % !!!Inhalt bei Fertigstellung der Arbeit l�schen
\newcommand{\workMarkDateTime}{}

% % Alle Namen werden im Titel und im hyperref-Paket eingetragen
% % !!! Ueberall f�r <Wert> das Entsprechende eintragen

 % <Typ> Studienarbeit, Dipolmarbeit, Studienarbeit oder Bachlor-Abschlussarbeit
\newcommand{\workTyp}{\workTodo{<Typ>}\xspace}

 % <Titel> der Arbeit
\newcommand{\workTitel}{Tracing-Tool zur Analyse von IO auf HPC-Systemen\xspace}

 % <Studiengang> z.B. Kommunikationstechnik
\newcommand{\workStudiengang}{\workTodo{<Studiengang>}\xspace}

% <Semester> mit Jahr z.B. Sommersemester 2008  
\newcommand{\workSemester}{\workTodo{<Semester>}\xspace}

% <Name> des Studenten
\newcommand{\workNameStudent}{\workTodo{<Name>}\xspace}

% <Pruefer> Name des pr�fenden (betreuenden) Professor an der Hochschule
\newcommand{\workPruefer}{Prof. Dr.-Ing. Rainer Keller\xspace} 


% %%% Nur bei Abschluss-Arbeiten

% <Datum> der Abgabe der Arbeit (Eidesstatliche Erkl�rung)
\newcommand{\workDatum}{\today\xspace}

% <Zweitpr�fer>
\newcommand{\workZweitPruefer}{\workTodo{<Zweitpr�fer>}\xspace}

% <Zeitraum>
\newcommand{\workZeitraum}{\workTodo{<Zeitraum>}\xspace}


% %%% Nur bei Industrie-Arbeiten:

% <Firma>
\newcommand{\workFirma}{\workTodo{<Firma>}\xspace}

% <Betreuer in der Firma>
\newcommand{\workBetreuer}{\workTodo{<Betreuer in der Firma>}\xspace}

% Firmenlogo Name hier anpassen, Gr��e (wenn m�glich) nicht �ndern
\newcommand{\workFirmenLogo}{\includegraphics[width=3cm]{fig/aa-titel/bwhpc}} 
